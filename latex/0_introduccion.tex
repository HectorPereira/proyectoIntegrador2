\section{Introducción}
\subsection{Planteamiento del problema}
La enseñanza de robótica manipuladora en contextos educativos (escuelas, UTU, liceo y cursos introductorios universitarios) suele verse limitada por el alto costo, riesgos de seguridad y poca “apertura” de los equipos comerciales. Esto reduce la práctica real de manipulación, dificulta integrar contenidos de mecánica, electrónica y control, y hace inviable llevar equipos a giras didácticas.

\subsection{Soluciones consideradas}
A partir del planteamiento del problema, se evaluaron tres alternativas tecnológicas para el accionamiento del brazo. A continuación se presentan de forma breve sus características, ventajas y limitaciones principales.

\begin{itemize}
  \item Servomotores (MG996R): Proporciona control directo por PWM, integración sencilla con microcontroladores y velocidad adecuada para demostraciones didácticas. En comparación con MG90 (insuficientes en par), los MG996R cubren el requerimiento de carga (hasta 100\,g a 40\,cm) con una repetibilidad aceptable. Ventajas: simplicidad, menor complejidad electrónica, buena disponibilidad y costo moderado. Consideraciones: picos de corriente, holguras propias del servo y límite de torque cerca del alcance máximo.
  
  \item Actuadores lineales tornillo---tuerca (motor DC con reductora). Ofrecen alta fuerza y buen costo, pero el mecanismo es más lento y con menor precisión sin realimentación. Requiere instrumentación adicional (encoders o potenciómetros, finales de carrera) y control más complejo para conocer el ángulo articular. Además, el juego mecánico (backlash) y la fricción pueden degradar la repetibilidad.
  
  \item Actuadores lineales con asistencia hidráulica (jeringas). Mejoran la relación fuerza---peso y permiten componentes económicos, pero añaden otra capa de complejidad: menor velocidad, histéresis, posibles fugas, purgado/cebado del circuito y mantenimiento. El diseño y control (válvulas, amortiguación) incrementan el esfuerzo de ingeniería frente a los beneficios en este contexto didáctico.
\end{itemize}


\begin{table}[h!]
\centering
\begin{tabular}{l c c c c}
\toprule
\textbf{Criterio} & \textbf{Peso} & \textbf{Servo} & \textbf{Actuador} & \textbf{Hidráulico} \\
\midrule
Peso        & 0.10 & 4 & 3 & 3 \\
Velocidad   & 0.20 & 4 & 2 & 1 \\
Simplicidad & 0.30 & 5 & 2 & 1 \\
Precio      & 0.20 & 4 & 4 & 4 \\
Fuerza      & 0.15 & 3 & 4 & 5 \\
Libertad    & 0.05 & 5 & 3 & 3 \\
\midrule
\textbf{Total ponderado} &       & \textbf{4.20} & \textbf{2.85} & \textbf{2.50} \\
\bottomrule
\end{tabular}
\caption{Matriz de selección de actuadores (1 = peor, 5 = mejor). Totales calculados con los pesos indicados.}
\end{table}

\newpage

\subsubsection*{Decisión técnica}
Se adopta la solución con \textbf{servomotores MG996R} por ofrecer el mejor equilibrio entre \emph{simplicidad}, \emph{velocidad} y \emph{libertad de movimiento}, manteniendo el \emph{costo} dentro del objetivo y una \emph{fuerza} suficiente para las tareas de \emph{pick \& place} previstas. Las otras alternativas se documentan como trabajo futuro para escenarios donde la fuerza disponible sea prioritaria sobre la velocidad o la sencillez de control.


\subsection{Solución propuesta}
Un brazo robótico didáctico a escala de 4 GDL con efector intercambiable: electroimán (para piezas ferromagnéticas) y garra (para piezas no magnéticas). Estructura paramétrica cortada en MDF (3.2–10 mm), tornillería estándar y electrónica de fácil reposición. Control con ESP32/ATmega328P, modos manual/automático, GUI en Python (Tkinter/PyQt), y documentación abierta para replicación.

\subsection{Objetivo General}
Diseñar, construir y validar un brazo robótico didáctico de 4 GDL, seguro, portable y de bajo costo, con efectores electroimán/garra intercambiables, capaz de ejecutar tareas básicas de pick \& place en un volumen de trabajo de hasta 50×50×50 cm.

\subsection{Objetivos específicos}
\begin{enumerate}
  \item \textbf{Repetibilidad:} $\leq$ 3--5\,mm en el volumen útil (validado con 10 repeticiones por punto).
  \item \textbf{Carga y alcance:} manipular $\leq$ 100\,g a un alcance de 40\,cm con tasa de éxito $\geq 90\%$ (10 intentos).
  \item \textbf{Tiempo de ciclo:} \emph{pick} $\rightarrow$ \emph{place} $\rightarrow$ retorno $\leq$ 3\,s a 10--15\,cm entre posiciones.
  \item \textbf{Efector intercambiable:} cambio imán/garra en $\leq 60$\,s sin herramientas especiales.
  \item \textbf{Seguridad:} paro de emergencia, cableado protegido, tensión segura en zona de usuario; checklist previo a operación.
  \item \textbf{Software:} GUI con lectura de posición, control ON/OFF del imán, modos manual/automático y cinemática inversa planar.
  \item \textbf{Documentación:} manual de armado, guía docente y 5 prácticas con rúbrica (pre/post test con mejora $\geq 20\%$).
\end{enumerate}
