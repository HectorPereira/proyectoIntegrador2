\section{Introducción}
\subsection{Planteamiento del problema}
La enseñanza de robótica manipuladora en contextos educativos (escuelas, UTU, liceo y cursos introductorios universitarios) suele verse limitada por el alto costo, riesgos de seguridad y poca facilidad de modificación y/o reparación de los equipos comerciales. Esto reduce la práctica real de manipulación, dificulta integrar contenidos de mecánica, electrónica y control, y hace inviable llevar equipos a giras didácticas.


\subsection{Solución propuesta}
Un brazo robótico didáctico a escala de 4 GDL con efector intercambiable: electroimán (para piezas ferromagnéticas) y garra (para piezas no magnéticas). Estructura paramétrica cortada en MDF (3.2–10 mm), tornillería estándar y electrónica de fácil reposición. Control con ESP32/ATmega328P, modos manual/automático, GUI en Python (Tkinter/PyQt), y documentación abierta para replicación.

\subsection{Objetivo General}
Diseñar, construir y validar un brazo robótico didáctico de 4 GDL, seguro, portable y de bajo costo, con efectores electroimán/garra intercambiables, capaz de ejecutar tareas básicas de pick \& place en un volumen de trabajo de hasta 50×50×50 cm.

\subsection{Objetivos específicos}
\begin{enumerate}
  \item \textbf{Repetibilidad:} $\leq$ 3--5\,mm en el volumen útil (validado con 10 repeticiones por punto).
  \item \textbf{Carga y alcance:} manipular $\leq$ 100\,g a un alcance de 40\,cm con tasa de éxito $\geq 90\%$ (10 intentos).
  \item \textbf{Tiempo de ciclo:} \emph{pick} $\rightarrow$ \emph{place} $\rightarrow$ retorno $\leq$ 3\,s a 10--15\,cm entre posiciones.
  \item \textbf{Efector intercambiable:} cambio imán/garra en $\leq 60$\,s sin herramientas especiales.
  \item \textbf{Seguridad:} paro de emergencia, cableado protegido, tensión segura en zona de usuario; checklist previo a operación.
  \item \textbf{Software:} GUI con lectura de posición, control ON/OFF del imán, modos manual y automático y cinemática inversa planar.
\end{enumerate}

\subsection{Obstáculos}
A partir del alcance y de las decisiones técnicas, se identifican los siguientes obstáculos que pueden afectar cronograma, costo y calidad del prototipo:

\begin{itemize}
  \item \textbf{Presupuesto:} mantener un costo \emph{razonable y no excesivo} considerando servos, fuente, electrónica y materiales.\ \emph{Mitigación:} priorizar compras esenciales, consolidar pedidos, alternativas locales y revisión de la BOM por etapas.
  
  \item \textbf{Tiempo de trabajo:} limitaciones de calendario académico y disponibilidad del equipo para CAD, corte, montaje y pruebas.\ \emph{Mitigación:} hitos semanales, versiones incrementales y criterios de aceptación por etapa.
  
  \item \textbf{Complejidad de diseño:} ajustes paramétricos para distintos espesores, transmisión por eslabones y tolerancias de corte.\ \emph{Mitigación:} plantillas de agujeros unificadas, librería de piezas parametrizadas y pruebas de “cupón” antes del corte final.
  
  \item \textbf{Fuerza de trabajo:} cantidad de integrantes y experiencia en mecánica, electrónica y software.\ \emph{Mitigación:} división clara de roles, documentación mínima obligatoria y revisiones cruzadas.
  
  \item \textbf{Disponibilidad de materiales:} variación de espesores de MDF (3.2--10\,mm), alternativas en acrílico/metal y stock de tornillería.\ \emph{Mitigación:} diseño con holguras controladas, insertos/espaciadores y piezas “shim” para compensar espesores.
  
  \item \textbf{Reparabilidad:} acceso a componentes para reemplazo, desgaste en articulaciones y cableado.\ \emph{Mitigación:} modularidad por subconjuntos, tornillería estándar M3, bujes/arandelas en puntos de fricción y canalización de cables.
  
  \item \textbf{Ensamblaje:} alineación de ejes, juego mecánico y centrado previo de servos.\ \emph{Mitigación:} guías paso a paso, marcas de referencia, útiles simples de alineación y checklist de calibración inicial.
\end{itemize}



