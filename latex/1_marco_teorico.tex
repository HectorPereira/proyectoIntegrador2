\section{Fundamento técnico-conceptual}

\subsection{Análisis de mercado}
Se realiza un periodo de investigación donde se recopilan modelos ya existentes en el mercado actual para solventar la problemática proporcionada. 

\subsubsection{GrabCAD}
GrabCAD es una plataforma comunitaria donde ingenieros y diseñadores comparten modelos 3D, ensamblajes y recursos CAD. Permite subir, buscar, previsualizar y descargar modelos en formatos comunes (p.\,ej.\ STEP/IGES, STL, SLDPRT), organizar colecciones y discutir mejoras en comentarios. Suele incluir descripciones, imágenes renderizadas y, en algunos casos, archivos auxiliares (planos o BOM).

En estas se encontraron múltiples diseños para brazos robóticos, como los que se pueden ver a continuación


\subsection{Soluciones consideradas}
A partir del planteamiento del problema, se evaluaron tres alternativas tecnológicas para el accionamiento del brazo. A continuación se presentan de forma breve sus características, ventajas y limitaciones principales.

\begin{itemize}
  \item \textbf{Servomotores (MG996R).} Proporcionan control directo por PWM, integración sencilla con microcontroladores y velocidad adecuada para demostraciones didácticas. En comparación con MG90 (insuficientes en par), los MG996R cubren el requerimiento de carga (hasta 100\,g a 40\,cm) con una repetibilidad aceptable.\ \emph{Detractor principal:} es la alternativa más cara dentro de las consideradas; se estimó un costo de \textbf{4 servos por \mbox{\$U~2600}} (disponibles en línea), aunque mantiene baja la complejidad de diseño y puesta en marcha.

  \item \textbf{Actuadores lineales tornillo---tuerca (motor DC con reductora).} Ofrecen alta fuerza y buen costo por actuador, y a niveles educativos más avanzados pueden resultar más \emph{didácticos} por la necesidad de instrumentación. Sin embargo, el mecanismo es más lento y menos preciso sin realimentación; requiere encoders o potenciómetros y finales de carrera para conocer el ángulo articular. Además, el diseño de eslabones y articulaciones se complica para alojar el husillo, y los mecanismos de codificación de posición son más propensos a \emph{interferencias/ruido} y se convierten en otro posible punto de fallo. El juego mecánico (backlash) y la fricción pueden degradar la repetibilidad.

  \item \textbf{Actuadores lineales con asistencia hidráulica (jeringas).} Es la alternativa más compleja: añade una capa de transmisión hidráulica sobre el actuador lineal. A cambio, ofrece una \emph{relación fuerza---peso} muy favorable (bajo peso en el extremo móvil frente a la fuerza aplicable). No obstante, introduce menor velocidad, histéresis, necesidad de purgado/cebado, riesgo de fugas y mayor mantenimiento; el control (válvulas, amortiguación) eleva la complejidad de ingeniería. Al igual que en la opción de tornillo---tuerca, la posición debe codificarse (p.\,ej., potenciómetros/encoders lineales), añadiendo sensores y puntos de fallo adicionales.
\end{itemize}


\begin{table}[h!]
\centering
\begin{tabular}{l c c c c}
\toprule
\textbf{Criterio} & \textbf{Peso} & \textbf{Servo} & \textbf{Actuador} & \textbf{Hidráulico} \\
\midrule
Peso        & 0.10 & 4 & 3 & 3 \\
Velocidad   & 0.20 & 4 & 2 & 1 \\
Simplicidad & 0.30 & 5 & 2 & 1 \\
Precio      & 0.20 & 4 & 4 & 4 \\
Fuerza      & 0.15 & 3 & 4 & 5 \\
Libertad    & 0.05 & 5 & 3 & 3 \\
\midrule
\textbf{Total ponderado} &       & \textbf{4.20} & \textbf{2.85} & \textbf{2.50} \\
\bottomrule
\end{tabular}
\caption{Matriz de selección de actuadores (1 = peor, 5 = mejor). Totales calculados con los pesos indicados.}
\end{table}

\newpage

\subsubsection*{Decisión técnica}
Se implementa la solución con \textbf{servomotores MG996R} por simplicidad de control, velocidad adecuada y disponibilidad, manteniendo el costo dentro del objetivo y una fuerza suficiente para las tareas de \emph{pick \& place}. 
No obstante, se adopta la filosofía de diseño observada en las alternativas de actuadores lineales e hidráulicos: 
\emph{concentrar masa cerca de los ejes de rotación} y \emph{transmitir el movimiento mediante eslabones} hasta las articulaciones distales. 
Con ello se reduce el brazo de palanca efectivo sobre cada servo, disminuye el par exigido y se baja la inercia en el extremo móvil, 
conservando parte de los beneficios de aquellas soluciones sin incorporar su complejidad de sensado y control.

\paragraph{Lineamientos de diseño adoptados}
\begin{itemize}
  \item Ubicar los actuadores lo más cerca posible de los ejes principales o de la base para minimizar el par requerido.
  \item Emplear eslabones/bielas para accionar articulaciones alejadas, manteniendo baja la masa en el efector.
  \item Limitar configuraciones que generen grandes brazos de palanca y añadir topes/curvas de velocidad para proteger a los servos.
  \item Mantener la \emph{modularidad} e \emph{intercambiabilidad} del efector (electroimán/garra) y el diseño paramétrico para distintos espesores de MDF.\@
\end{itemize}

En síntesis, el resultado es un diseño híbrido control directo con servos potentes y una arquitectura mecánica inspirada en sistemas lineales/hidráulicos que traslada masa hacia la estructura fija. 
Esto simplifica la implementación y el mantenimiento, a la vez que mejora la eficiencia mecánica y la repetibilidad del prototipo.

\newpage
\section{Alcance}

\subsection{Alcance del proyecto}
El proyecto comprende el diseño, construcción y validación de un brazo robótico didáctico a escala. La arquitectura mecánica contempla cuatro grados de libertad (sin contar el efector), con una estructura paramétrica diseñada para corte láser en MDF de espesores entre 3.2 y 10 mm, pero adaptable también a materiales como acrílico o metal. La base será fija y deberá estar contenida dentro de un volumen de 50 × 50 × 50 cm.

El sistema contará con efectores intercambiables, entre ellos un electroimán para la manipulación de piezas ferromagnéticas y una garra mecánica para piezas no magnéticas. Ambos estarán diseñados con un mecanismo de cambio rápido que permita alternar entre uno y otro sin dificultad.

La actuación del brazo se logrará mediante servomotores MG996R, con uno por cada articulación principal. Para optimizar el rendimiento, se implementará una transmisión por eslabones que mantenga la mayor parte de la masa cercana a los ejes de giro.

En cuanto al control y la electrónica, se utilizará un microcontrolador (ESP32 o ATmega328P), un driver MOSFET como el IRFZ44N (o equivalente) para gobernar el electroimán, así como finales de carrera para homing y elementos de señalización visual y sonora mediante LEDs y un buzzer de estado.

El software incluirá un firmware desarrollado en C/C++ y una interfaz gráfica en Python, implementada en Tkinter o PyQt. Esta ofrecerá un modo manual y un modo automático, además de control ON/OFF del electroimán y una cinemática inversa planar básica para la manipulación de piezas.

La alimentación del sistema se resolverá con una fuente capaz de abastecer tanto la lógica como la potencia, incorporando convertidores adecuados para los servos y el electroimán.

Finalmente, en el componente de documentación y docencia, se elaborará un manual de armado, una guía docente y un conjunto mínimo de cinco prácticas de laboratorio orientadas a la enseñanza y experimentación con el brazo robótico.


\subsection{Fuera de alcance}
Se considera fuera de alcanze una versión implementada con actuadores lineales o hidráulicos (solo se documentan como alternativas). Además, incorporar algún tipo de visión artificial o seguimiento por cámara con planificación avanzada de trayectorias en 3D. Finalmente, se considera fuera de alcance la calibración metrológica de alta precisión o certificaciones industriales.


\subsection{Criterios de aceptación}
El dispositivo debe ser capaz de ejecutar trayectorias punto a punto tanto en modo manual como en modo automático. Asimismo, deberá alcanzar una tasa de éxito superior al 90 \% en las tareas de agarre y manipulación de objetos, garantizando un desempeño confiable durante su operación.

De igual forma, contará con una rutina de referencia operativa que asegure la correcta inicialización del sistema, un mecanismo de paro de emergencia y límites mecánicos verificados para preservar la seguridad del equipo y del entorno.

Finalmente, el proyecto se entregará con un prototipo completamente ensamblado, acompañado del firmware correspondiente, la interfaz gráfica de usuario, un manual de usuario detallado y una guía docente que incluya cinco prácticas evaluables para su aplicación en el ámbito académico.